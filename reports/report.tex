\documentclass[a4paper,11pt]{article}
\usepackage{fullpage}
\usepackage{hyperref}
\usepackage{url}
\usepackage{graphicx}
\usepackage{polski}
\usepackage[utf8]{inputenc}

\setlength{\parindent}{0pt}
\addtolength{\parskip}{\baselineskip}

\title{3rd Year Group Project\\Report One\\\emph{Inception}}
% \author{Maciej Albin, Rafal Szymanski, Sam Wong, Suhaib Sarmad, Jamal Khan, Suhaib Sarmad}

\author{
    \small{Rafał Szymański}\\
%		\small{\emph{rs2909}}
  	\and
    \small{Maciek Albin}\\
%		\small{\emph{mja108}}
    \and
    \small{Sam Wong}\\
%		\small{\emph{sw2309}}
    \and
    \small{Suhaib Sarmad}\\
%		\small{\emph{sss308}}
		\and
		\small{Jamal Khan}\\
%		\small{\emph{jzk109}}
		\and
		\small{\{rs2909, mja108, sw2309, sss308, jzk09\}@doc.ic.ac.uk}
		\and
		\\Department of Computing - Imperial College London
}

\date{}


\begin{document} 
	\maketitle
	%\tableofcontents
	%\newpage
	
	\section{Introduction}
	
	\begin{quote}
\emph{In a world where technology exists to enter the human mind through dream invasion, a highly skilled thief is given a final chance at redemption which involves executing his toughest job to date: Inception.}	\end{quote}

	In this group project, we wish to provide an interface to let people see the public Twitter reaction to current events sourced from news RSS feeds.

	
	\section{Key Requirements}
		
		\begin{itemize}
			\item Getting top news stories from Google News RSS Feed
			\item Generating keywords from those news stories, including headlines, and the articles themselves
			\item Getting relevant tweets using the Twitter Streaming API, and giving it the keyword list above
			\item Attaching tweets to appropriate news stories
			\item Generating word clouds for news stories
			\item Fluid live updating UI
		\end{itemize}
		
	\section{Extensions}
	
		\begin{itemize}
			\item Sentiment Analysis of the tweets
			\item Categorisation of tweets into topics
			\item Story images from either the article itself, or other sources
		\end{itemize}
	
	\section{Choice of Development Method}
		\subsection{Development Methods}
		
			We are planning on using agile development methodology with scrum framework. As our development board that has our \emph{product backlog}, and the \emph{current iteration}, we shall be using Trello\footnote{\url{http://trello.com}}.
			\begin{center}
			\includegraphics[scale=0.4]{trello.png}
		  \end{center}
	
			The \emph{product backlog} has all the features that we would like to implement. This list will have items added to it as the project progresses. At the beginning of each week, we shall have a scrum meeting where we pull some required features from the \emph{product backlog} into an \emph{iteration list} that will keep track of what is to be done each week. The Trello software allows us to assign developers to particular tasks, and this feature will be used.
			
			We shall have scrum meetings either daily, or every two days, where we will discuss the progress of the task everyone has been assigned to for that week. We aim to successfully complete the required tasks at the end of each iteration period. This is how we will measure our velocity. If we are not getting everything done from the current period, it means we have not achieved proper velocity.
			
			For development itself we will try to utilise \emph{paired programming} as much as possible. We believe this will allow all members of the team to understand major parts of the application in detail and help to even out different levels of expertise between team members. We also believe it will boost the team's productivity.
		
		\subsection{Team Roles}
		  \subsubsection{Scrum}
		    Scrum framework contains 3 main roles:
		    \begin{description}
		      \item[Scrum Master] is responsible for maintaining the process. This role is held by \emph{Rafał Szymański}.
		      \item[Product Owner] is responsible for the feature list. This role is held by \emph{Maciek Albin}.
		      \item[The Team] is responsible for developing the software. We are all members of the team.
		    \end{description} 
		    
		  \subsubsection{Development}
		    We have to key areas we've divided the team into, but there will definitely be some migrations between them later in the project:
		    \begin{description}
		      \item[Frontend design group] is responsible for creating a fast, fluid and graphically stimulating UI for the website that will communicate with the backend systems. \emph{Sam Wong} and \emph{Suhaib Sarmad} are responsible for this.
		      \item[Backend design group] is responsible for developing a fast backend system that will periodically analyse news stories and provide an API for the javascript frontend. \emph{Jamal Khan}, \emph{Rafał Szymański} and \emph{Maciek Albin} are members of this group.
		    \end{description}
	
		\subsection{Version Control and Code Testing}
		
			We will be using git for version control. Currently the project is hosted on github\footnote{\url{https://github.com/radicality/Twitter-News}}, but we will probably move it soon to our own VPS to set up continuos integration, and instant deployment.
			
			We will be using a unit testing framework to check that the code we are writing meets its specification. This will be done using the PyUnit\footnote{\url{http://pyunit.sourceforge.net/}} testing framework.
			
			
		\subsection{Needed Information Technology}
			We have decided to use Python as the core programming language, along with MongoDB as the persistent datastore for keeping news and tweet information. MongoDB is particularly useful, as it allows easy saving and retrieval of tweets that come in json format and as opposed to a rigid SQL database it allows us to store `documents'. For the UI we'll be using HTML5 and JavaScript.
			
	
	\section{Draft Schedule}
	
		We are currently in week 4 of the term, and the project is to be handed in by 10th December, i.e the end of week 10.
		
		Our progression plan is as follows:
		
		\begin{itemize}
			\item Week 1: Setup DB and server to receive Tweets and RSS, UI Mockups
			\item Week 2: UI Implementation, Design algorithms to generate keywords
			\item Week 3: Getting the web app online (Beta, Iteration 1)
			\item Week 4: Debugging, refactoring and optimization (Iteration 2)
			\item Week 5: Work on Extensions, Sentiment analysis algorithm, Categorization algorithm (Iteration 3)
			\item Week 6: Final polishing of app (Iteration 4)
		\end{itemize}
	
	\section{Outline of different iterations}
	\begin{description}
	  \item[First Iteration] MVP - basic news augmenting functionality and live updates.
	    \begin{itemize}
	      
	      \item Setup DB and server to receive Tweets and RSS
	      
	      \begin{itemize}
	        \item Research suitable DB (MongoDB)
	        \item Select frameworks and architecture
	        \item digging into twitter API \& RSS
	      \end{itemize}
	      
	      \item Design algorithms to generate keywords (Look into using natural language processing libraries, there are a lot available for Python)
	      \item UI Mockups and UX Research
	      \item UI Implementation: Framework and plugins to use (eg jQuery, jQuery plugins, Twitter Bootstrap, etc)
	      \item Getting the web app hosted
	    \end{itemize}
	    \item[Second Iteration] 
	    Polished and Efficient implementation including Debugging, refactoring and optimisation
	    \item[Third Iteration] 
	    Extensions including:
	      \begin{itemize}
	        \item Sentiment Analysis algorithm
	        \item Categorization algorithm    
	      \end{itemize}
	      
	    \item[Fourth Iteration] 
	    Final Product
	  
	  
	\end{description}
	
	\section{Detailed Plan for First Iteration}
	\begin{itemize}
	  \item Create Dummy API server for the UI to connect to and define API calls
	  \item Basic UI for displaying news stories
	  \item Create a tweet fetching and analysis process:
	  \begin{itemize}
	    \item Thread to extract keywords from google news stories
	    \item Thread to fetch tweets based on keywords
	    \item Thread to assign tweets to news stories
	  \end{itemize}
	  \item Setup continuos integration
	\end{itemize}
	
	We have 3 weeks = 15 working days for 5 day workweek. Our first iteration plan is:
	
	\begin{itemize}
		\item Design system architecture (day 1-3)
		\item Setup continuous integreation, version control, testing suites (day 1-3)
		\item Create a dummy API server for the UI to connect to (day 4)
		\item Tweet fetching thread (day 5-10)
		\item Create a keyword extractor from google news rss thread (day 5-10)
		\item Analysis thread that scans current news list and creates the initial news stories in DB (day 11-13)
		\item Matching Tweets to News (day 11-13)
		\item UI Mockups (day 4-6)
		\item UI debates (day 7)
		\item UI refinements (day 8)
		\item UI Implementation (day 9-13)
		\item Integrate UI with system (day 9-13)
		\item Finalise \& polish first iteration (day 14-15)
	\end{itemize}
	
	Progress Measure:
	
	\begin{itemize}
		\item Day 3: Architecture \& Design Finalised
		\item Day 5: Dummy API exists so UI \& backend can be developed concurrently
		\item Day 10: Datagathering systems all in place
		\item Day 13: UI functional, News-Tweets Paired
		\item Day 15: MVP done
	\end{itemize}
	
	Potential Risks:
	
	\begin{itemize}
		\item Failure to develop keyword algorithm in time
		\item Small amount of tweets for a news story
		\item DB grows too quickly / run out of space / efficiency problems.
	\end{itemize}


	
	
		

\end{document}